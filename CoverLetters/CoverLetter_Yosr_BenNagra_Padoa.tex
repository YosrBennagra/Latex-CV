\documentclass[11pt,a4paper]{article}

\newif\ifaccentcolor
\accentcolorfalse 

% -------- Document Base Packages --------
\usepackage[utf8]{inputenc} % Character encoding
\usepackage[T1]{fontenc}    % Better hyphenation & copyable text
\usepackage{lmodern}        % Latin Modern - clean, professional font
\usepackage{microtype}      % Better kerning/justification without odd spacing
\usepackage[english]{babel} % Language metadata and hyphenation patterns
\usepackage[margin=0.5in, top=0.45in, bottom=0.45in]{geometry} % Tight margins for 1 page
\usepackage{setspace} % For fine line spacing control if needed
\usepackage{enumitem} % Tighter itemize environments
\usepackage{hyperref} % Clickable links (safe for ATS)
\usepackage{url}
\urlstyle{same} % Keep URLs in the same font for readability/ATS
\hypersetup{
  colorlinks=true,
  urlcolor=black,
  linkcolor=black,
  pdflang={en},
  pdfauthor={Your Name},
  pdfproducer={},
  pdftitle={Curriculum Vitae},
  pdfsubject={Professional CV},
  pdfkeywords={CV, Resume}
}

% Ensure Unicode text is copyable in the generated PDF (ATS-friendly)
% Only if the file exists in the TeX installation
\IfFileExists{glyphtounicode.tex}{\input{glyphtounicode}\pdfgentounicode=1}{}
\usepackage{xcolor}
\ifaccentcolor
  \definecolor{Accent}{HTML}{003366}
\else
  \definecolor{Accent}{gray}{0}
\fi

% Avoid extra spacing after periods (American style acceptable)
\frenchspacing

% -------- Spacing Tweaks --------
% Tight spacing for single page
\setlength{\parskip}{0pt}
\setlength{\parindent}{0pt}

% Consistent list formatting for bullets - very compact
\setlist[itemize]{
  leftmargin=8pt,
  itemsep=0pt,
  topsep=0pt,
  parsep=0pt,
  partopsep=0pt
}

% -------- Helper Macros (public, ATS-safe) --------
% Use public macro names (no @) to avoid needing \makeatletter.
\newcommand{\cvname}[1]{\def\CVName{#1}}
\newcommand{\cvrole}[1]{\def\CVRole{#1}}
\newcommand{\cvcontact}[1]{\def\CVContact{#1}}
% Granular contact fields
\newcommand{\cvphone}[1]{\def\CVPhone{#1}}
\newcommand{\cvemail}[1]{\def\CVEmail{#1}}
\newcommand{\cvaddress}[1]{\def\CVAddress{#1}}
\newcommand{\cvgithub}[1]{\def\CVGithub{#1}}
\newcommand{\cvwebsite}[1]{\def\CVWebsite{#1}}
\newcommand{\cvsummary}[1]{\def\CVSummary{#1}}

% Provide defaults
\cvname{Your Name}
\cvrole{Professional Title / Target Role}
\cvcontact{City, ST | (555) 555-5555 | professional.email@example.com | LinkedIn: linkedin.com/in/username | GitHub: github.com/username}
\cvphone{+000 0000000}
\cvemail{your.email@example.com}
\cvaddress{City, Country}
\cvgithub{https://github.com/username}
\cvwebsite{https://portfolio.example.com}
\cvsummary{Impact-focused professional with X+ years of experience ... (2-3 concise, keyword-rich sentences highlighting scope, strengths, domains, and outcomes.)}

% Section heading macro (simple for ATS parsing)
\newcommand{\cvsection}[1]{%
  \vspace{3pt}\par\textbf{\large #1}\par\vspace{1.5pt}\hrule height 0.6pt \vspace{2pt}
}

% Header macro - name and role on same line, contact below
\newcommand{\makecvheader}{%
  \begingroup
  \par\vspace*{0pt}
  {\Large\textbf{\MakeUppercase{\CVName}}\hfill\Large\textbf{\CVRole}}\par
  \vspace{1pt}
  \endgroup
}

% Contact information - compact single line
\newcommand{\makecontactinfo}{%
  {\small
    \CVPhone~\textbar~\href{mailto:\CVEmail}{\CVEmail}~\textbar~\CVAddress~\textbar~\href{\CVGithub}{GitHub}~\textbar~\href{\CVWebsite}{Portfolio}%
  }\par\vspace{2pt}
}

% Internship environment - tight spacing for 1 page
% Usage: \begin{internship}{Title}{Company}{Location}{Dates}{Company description}
%           \item Task bullet ...
%       \end{internship}
\newenvironment{internship}[5]{%
  \par\textbf{#1}\hfill{\footnotesize\textit{#4}}\\[-2pt]
  #2\\[-5pt]
  \ifx&#5&\else{\footnotesize\textcolor{gray}{#5}}\\[-6pt]\fi
  \begin{itemize}[leftmargin=8pt,itemsep=0pt,topsep=0pt]
}{\end{itemize}\vspace{2.5pt}}

% Entry macro: Title | Company | Location | Dates
% Usage: \cventry{TITLE}{COMPANY}{LOCATION}{DATES}{CONTENT}
\newcommand{\cventry}[5]{%
  \textbf{#1} \hfill #4\\
  \textit{#2, #3}\\[-4pt]
  #5\vspace{6pt}
}

% Compact bullet alias (alternative inline bullet list if ever needed)
\newcommand{\cvbullet}[1]{\item #1}

% Keywords environment (comma-separated plain text for ATS)
\newenvironment{cvkeywords}{%
  \vspace{4pt}\noindent\begin{minipage}{\textwidth}\textbf{Keywords: }\begingroup\raggedright\small
}{\par\endgroup\end{minipage}\vspace{4pt}}

% Date range helper (ensures consistent style)
\newcommand{\daterange}[2]{#1 -- #2}

% (Removed custom \endash macro to avoid redefinition warnings; use -- directly for en-dash)

% Safety: ensure no widow lines at page start if possible
\clubpenalty=10000
\widowpenalty=10000
\hyphenpenalty=10000
\exhyphenpenalty=10000

% Page style: no headers/footers and no page numbers
\pagestyle{empty}

% Single page bias: tighten a bit if near overflow (user can adjust)
% Uncomment to shrink vertical space a little more:
% \setlength{\parskip}{1pt}
% \setlist[itemize]{itemsep=1pt, topsep=1pt}

% End preamble macros

% Summary macro (renders the CV summary text if provided)
\newcommand{\makecvsummary}{{\small \CVSummary}\par}

% Defer metadata that uses user-provided macros until after they are defined
% so the PDF properties match the actual name/role/keywords.
\AtBeginDocument{%
  \hypersetup{
    pdfauthor={\CVName},
    pdftitle={\CVName\space — \CVRole},
    pdfsubject={Resume / CV},
    pdfkeywords={Software, Full-Stack, React, NestJS, Spring Boot, JavaScript, TypeScript, Python, MongoDB, SQL, DevOps, CI/CD, Docker, Jenkins, RAG}
  }
}


\begin{document}

\begin{center}
    {\LARGE Lettre de Motivation}\\[4pt]
    {\large Développeur FullStack IA}
\end{center}
\vspace{10pt}

Ariana, Tunisie\\
25 novembre 2025\\[6pt]

Équipe de Recrutement\\
Padoa\\
Paris, France\\[10pt]

Chère Équipe de Recrutement,

Je vous écris pour exprimer mon vif intérêt pour le poste de Développeur FullStack IA au sein de la squad spécialisée dans l'intelligence artificielle chez Padoa. En tant qu'ingénieur fraîchement diplômé avec une expérience pratique en intégration de LLMs et en construction de systèmes RAG, je suis enthousiaste à l'idée de contribuer à votre mission de moderniser la santé au travail et d'améliorer la vie de millions de salariés en France. Votre approche combinant technologie, data et impact social résonne profondément avec mes valeurs et mes aspirations professionnelles.

Durant mon stage de fin d'études chez IT Serv, j'ai conçu et développé une plateforme web full-stack intégrant des fonctionnalités IA basées sur l'architecture RAG (Retrieval-Augmented Generation). J'ai fine-tuné des modèles de langage avec Hugging Face, implémenté un vérificateur de symptômes alimenté par l'IA, et construit des APIs REST pour connecter un frontend React avec des services backend en Flask et MongoDB. Ce travail m'a permis d'acquérir une compréhension approfondie du prompt engineering, de l'optimisation des embeddings, et du monitoring des performances des modèles en production (qualité des réponses, latence, coût). J'ai également mis en place des pipelines CI/CD avec Docker et Jenkins pour assurer des déploiements fiables. Chez IronByte, j'ai développé une application éducative avec NestJS (TypeScript) et React, en mettant l'accent sur une architecture propre et des tests robustes. Plus tôt, chez Ooredoo, j'ai construit une application de communication interne utilisant Spring Boot (Java) et Angular, avec des fonctionnalités temps réel et une couverture complète de tests unitaires et d'intégration.

Ma stack technique actuelle inclut Python, Java, TypeScript/JavaScript, React, Angular, NestJS, Spring Boot, Flask, ainsi que MongoDB, PostgreSQL et MySQL. Je maîtrise les frameworks modernes (Angular, React), les bases de données relationnelles et NoSQL, et j'ai de solides notions d'architecture technique et d'optimisation de code. Mon expérience avec Hugging Face, les modèles de langage, et les architectures RAG me permet de contribuer immédiatement à vos projets d'IA générative. Je suis également habitué à travailler en environnement Agile, à participer aux code reviews, et à partager mes connaissances avec l'équipe. Je suis motivé par la production de code maintenable, bien testé (unitaire, intégration, end-to-end), et conforme aux bonnes pratiques de développement.

Ce qui m'attire particulièrement chez Padoa, c'est votre engagement envers l'impact positif, votre environnement technique de qualité avec plus de 50\% de profils tech, et votre volonté de faire évoluer vos collaborateurs sur le long terme. Je suis enthousiaste à l'idée de développer des solutions d'IA générative qui améliorent concrètement la santé au travail, de participer aux choix d'architecture et de modèles, et de collaborer avec une équipe passionnée et pédagogue. Votre stack technique (Python, Node.js + Express + TypeScript, PostgreSQL, Angular 18) correspond parfaitement à mes compétences et à mes intérêts.

Vous pouvez découvrir mes projets, mon parcours technique et des démos en direct sur mon portfolio en ligne : \url{https://portfolio-yosr.vercel.app/en}.

Je vous remercie de l'attention portée à ma candidature. Je serais ravie de discuter de la manière dont mon expérience en développement full-stack, mon expertise en LLMs et systèmes RAG, et ma passion pour l'IA générative peuvent contribuer à la mission de Padoa et aider à transformer le monde de la santé au travail.

Cordialement,\\[4pt]
Yosr Ben Nagra

\end{document}
