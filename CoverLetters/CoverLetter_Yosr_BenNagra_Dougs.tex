\documentclass[11pt,a4paper]{article}

\newif\ifaccentcolor
\accentcolorfalse 

% -------- Document Base Packages --------
\usepackage[utf8]{inputenc} % Character encoding
\usepackage[T1]{fontenc}    % Better hyphenation & copyable text
\usepackage{lmodern}        % Latin Modern - clean, professional font
\usepackage{microtype}      % Better kerning/justification without odd spacing
\usepackage[english]{babel} % Language metadata and hyphenation patterns
\usepackage[margin=0.5in, top=0.45in, bottom=0.45in]{geometry} % Tight margins for 1 page
\usepackage{setspace} % For fine line spacing control if needed
\usepackage{enumitem} % Tighter itemize environments
\usepackage{hyperref} % Clickable links (safe for ATS)
\usepackage{url}
\urlstyle{same} % Keep URLs in the same font for readability/ATS
\hypersetup{
  colorlinks=true,
  urlcolor=black,
  linkcolor=black,
  pdflang={en},
  pdfauthor={Your Name},
  pdfproducer={},
  pdftitle={Curriculum Vitae},
  pdfsubject={Professional CV},
  pdfkeywords={CV, Resume}
}

% Ensure Unicode text is copyable in the generated PDF (ATS-friendly)
% Only if the file exists in the TeX installation
\IfFileExists{glyphtounicode.tex}{\input{glyphtounicode}\pdfgentounicode=1}{}
\usepackage{xcolor}
\ifaccentcolor
  \definecolor{Accent}{HTML}{003366}
\else
  \definecolor{Accent}{gray}{0}
\fi

% Avoid extra spacing after periods (American style acceptable)
\frenchspacing

% -------- Spacing Tweaks --------
% Tight spacing for single page
\setlength{\parskip}{0pt}
\setlength{\parindent}{0pt}

% Consistent list formatting for bullets - very compact
\setlist[itemize]{
  leftmargin=8pt,
  itemsep=0pt,
  topsep=0pt,
  parsep=0pt,
  partopsep=0pt
}

% -------- Helper Macros (public, ATS-safe) --------
% Use public macro names (no @) to avoid needing \makeatletter.
\newcommand{\cvname}[1]{\def\CVName{#1}}
\newcommand{\cvrole}[1]{\def\CVRole{#1}}
\newcommand{\cvcontact}[1]{\def\CVContact{#1}}
% Granular contact fields
\newcommand{\cvphone}[1]{\def\CVPhone{#1}}
\newcommand{\cvemail}[1]{\def\CVEmail{#1}}
\newcommand{\cvaddress}[1]{\def\CVAddress{#1}}
\newcommand{\cvgithub}[1]{\def\CVGithub{#1}}
\newcommand{\cvwebsite}[1]{\def\CVWebsite{#1}}
\newcommand{\cvsummary}[1]{\def\CVSummary{#1}}

% Provide defaults
\cvname{Your Name}
\cvrole{Professional Title / Target Role}
\cvcontact{City, ST | (555) 555-5555 | professional.email@example.com | LinkedIn: linkedin.com/in/username | GitHub: github.com/username}
\cvphone{+000 0000000}
\cvemail{your.email@example.com}
\cvaddress{City, Country}
\cvgithub{https://github.com/username}
\cvwebsite{https://portfolio.example.com}
\cvsummary{Impact-focused professional with X+ years of experience ... (2-3 concise, keyword-rich sentences highlighting scope, strengths, domains, and outcomes.)}

% Section heading macro (simple for ATS parsing)
\newcommand{\cvsection}[1]{%
  \vspace{3pt}\par\textbf{\large #1}\par\vspace{1.5pt}\hrule height 0.6pt \vspace{2pt}
}

% Header macro - name and role on same line, contact below
\newcommand{\makecvheader}{%
  \begingroup
  \par\vspace*{0pt}
  {\Large\textbf{\MakeUppercase{\CVName}}\hfill\Large\textbf{\CVRole}}\par
  \vspace{1pt}
  \endgroup
}

% Contact information - compact single line
\newcommand{\makecontactinfo}{%
  {\small
    \CVPhone~\textbar~\href{mailto:\CVEmail}{\CVEmail}~\textbar~\CVAddress~\textbar~\href{\CVGithub}{GitHub}~\textbar~\href{\CVWebsite}{Portfolio}%
  }\par\vspace{2pt}
}

% Internship environment - tight spacing for 1 page
% Usage: \begin{internship}{Title}{Company}{Location}{Dates}{Company description}
%           \item Task bullet ...
%       \end{internship}
\newenvironment{internship}[5]{%
  \par\textbf{#1}\hfill{\footnotesize\textit{#4}}\\[-2pt]
  #2\\[-5pt]
  \ifx&#5&\else{\footnotesize\textcolor{gray}{#5}}\\[-6pt]\fi
  \begin{itemize}[leftmargin=8pt,itemsep=0pt,topsep=0pt]
}{\end{itemize}\vspace{2.5pt}}

% Entry macro: Title | Company | Location | Dates
% Usage: \cventry{TITLE}{COMPANY}{LOCATION}{DATES}{CONTENT}
\newcommand{\cventry}[5]{%
  \textbf{#1} \hfill #4\\
  \textit{#2, #3}\\[-4pt]
  #5\vspace{6pt}
}

% Compact bullet alias (alternative inline bullet list if ever needed)
\newcommand{\cvbullet}[1]{\item #1}

% Keywords environment (comma-separated plain text for ATS)
\newenvironment{cvkeywords}{%
  \vspace{4pt}\noindent\begin{minipage}{\textwidth}\textbf{Keywords: }\begingroup\raggedright\small
}{\par\endgroup\end{minipage}\vspace{4pt}}

% Date range helper (ensures consistent style)
\newcommand{\daterange}[2]{#1 -- #2}

% (Removed custom \endash macro to avoid redefinition warnings; use -- directly for en-dash)

% Safety: ensure no widow lines at page start if possible
\clubpenalty=10000
\widowpenalty=10000
\hyphenpenalty=10000
\exhyphenpenalty=10000

% Page style: no headers/footers and no page numbers
\pagestyle{empty}

% Single page bias: tighten a bit if near overflow (user can adjust)
% Uncomment to shrink vertical space a little more:
% \setlength{\parskip}{1pt}
% \setlist[itemize]{itemsep=1pt, topsep=1pt}

% End preamble macros

% Summary macro (renders the CV summary text if provided)
\newcommand{\makecvsummary}{{\small \CVSummary}\par}

% Defer metadata that uses user-provided macros until after they are defined
% so the PDF properties match the actual name/role/keywords.
\AtBeginDocument{%
  \hypersetup{
    pdfauthor={\CVName},
    pdftitle={\CVName\space — \CVRole},
    pdfsubject={Resume / CV},
    pdfkeywords={Software, Full-Stack, React, NestJS, Spring Boot, JavaScript, TypeScript, Python, MongoDB, SQL, DevOps, CI/CD, Docker, Jenkins, RAG}
  }
}


\cvname{Yosr Ben Nagra}
\cvrole{Software Engineer}
\cvcontact{Ariana, Tunisie}
\cvphone{+216 53916040}
\cvemail{yosrbennagra@gmail.com}
\cvaddress{Ariana, Tunisie}
\cvgithub{https://github.com/YosrBennagra}
\cvwebsite{https://portfolio-yosr.vercel.app/en}

\begin{document}
\makecvheader
\makecontactinfo

\vspace{12pt}

\noindent Chère équipe Dougs,

Je suis vraiment enthousiaste à l'idée de rejoindre Dougs et de contribuer à votre mission : simplifier la vie des entrepreneurs pour qu'ils réussissent. Votre approche disruptive qui allie conseil et tech résonne parfaitement avec ma vision du développement logiciel : créer des outils qui ont un impact réel et qui rendent la vie des gens plus simple.

J'ai une expérience concrète avec votre stack technique. Chez IronByte, j'ai développé une application éducative complète avec NestJS côté backend, en construisant des APIs RESTful robustes pour la gestion des devoirs, le partage de cours et la création d'emplois du temps. Chez Ooredoo Tunisie, j'ai travaillé avec Angular pour créer une application de communication interne avec chat en temps réel, filtrage et recherche. Cette combinaison NestJS + Angular me permet de comprendre les deux côtés de la stack et de concevoir des APIs que les équipes frontend adorent utiliser.

Je maîtrise SQL : j'ai travaillé avec PostgreSQL, MySQL, MongoDB et Oracle sur différents projets. Dans mon projet de plateforme collaborative (type Notion), j'ai utilisé NestJS avec TypeScript, WebSockets pour la collaboration en temps réel, et Jest pour les tests automatisés. J'ai aussi mis en place des pipelines CI/CD avec GitHub Actions et Jenkins, ce qui m'a appris l'importance de l'automatisation et de la qualité du code.

Je veux être transparente : j'ai 2 ans d'expérience avec Node.js et TypeScript plutôt que les 5 ans mentionnés dans l'offre. Cependant, pendant ces 2 années, j'ai travaillé intensivement sur des projets full-stack complexes, participé à des revues de code, écrit des tests unitaires et d'intégration, et collaboré étroitement avec des équipes multidisciplinaires. Je suis curieuse, force de proposition, et la qualité est au cœur de mes valeurs, exactement ce que vous recherchez.

Ce qui m'enthousiasme particulièrement chez Dougs, c'est votre utilisation de la méthode Shape Up pour redonner ownership aux développeurs. J'adore l'idée d'avoir un impact important sur le produit tout en travaillant sur des problématiques concrètes qui touchent à la fois les besoins métier et l'expérience développeur. Votre framework qui organise le travail des équipes métier semble être un défi technique passionnant.

Je suis très à l'aise pour communiquer avec différentes personnes de l'entreprise pour trouver des réponses. Je crois profondément que le relationnel, le travail d'équipe et le management positif sont les clés de la réussite. J'ai pratiqué Agile/Scrum dans mes projets, et je suis motivée pour approfondir mes connaissances en DDD et Clean Code, des pratiques que je trouve essentielles pour construire des applications scalables et maintenables.

Rejoindre la squad opérations de Kévin, Alexis, Nicolas, Thibaut et Célia serait une opportunité fantastique d'apprendre, de contribuer et de grandir. Je suis prête à participer au recrutement, faire de la veille technologique, et challenger le produit ainsi que le code à travers les reviews.

Merci d'avoir pris le temps de considérer ma candidature. Je serais ravie de discuter de comment mon expérience avec NestJS, Angular et ma passion pour la qualité peuvent contribuer à l'aventure Dougs.

\vspace{6pt}
\noindent Cordialement,\\
\textbf{Yosr Ben Nagra}

\end{document}
