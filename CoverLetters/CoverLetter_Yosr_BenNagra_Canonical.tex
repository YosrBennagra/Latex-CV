\documentclass[11pt,a4paper]{article}

\newif\ifaccentcolor
\accentcolorfalse 

% -------- Document Base Packages --------
\usepackage[utf8]{inputenc} % Character encoding
\usepackage[T1]{fontenc}    % Better hyphenation & copyable text
\usepackage{lmodern}        % Latin Modern - clean, professional font
\usepackage{microtype}      % Better kerning/justification without odd spacing
\usepackage[english]{babel} % Language metadata and hyphenation patterns
\usepackage[margin=0.5in, top=0.45in, bottom=0.45in]{geometry} % Tight margins for 1 page
\usepackage{setspace} % For fine line spacing control if needed
\usepackage{enumitem} % Tighter itemize environments
\usepackage{hyperref} % Clickable links (safe for ATS)
\usepackage{url}
\urlstyle{same} % Keep URLs in the same font for readability/ATS
\hypersetup{
  colorlinks=true,
  urlcolor=black,
  linkcolor=black,
  pdflang={en},
  pdfauthor={Your Name},
  pdfproducer={},
  pdftitle={Curriculum Vitae},
  pdfsubject={Professional CV},
  pdfkeywords={CV, Resume}
}

% Ensure Unicode text is copyable in the generated PDF (ATS-friendly)
% Only if the file exists in the TeX installation
\IfFileExists{glyphtounicode.tex}{\input{glyphtounicode}\pdfgentounicode=1}{}
\usepackage{xcolor}
\ifaccentcolor
  \definecolor{Accent}{HTML}{003366}
\else
  \definecolor{Accent}{gray}{0}
\fi

% Avoid extra spacing after periods (American style acceptable)
\frenchspacing

% -------- Spacing Tweaks --------
% Tight spacing for single page
\setlength{\parskip}{0pt}
\setlength{\parindent}{0pt}

% Consistent list formatting for bullets - very compact
\setlist[itemize]{
  leftmargin=8pt,
  itemsep=0pt,
  topsep=0pt,
  parsep=0pt,
  partopsep=0pt
}

% -------- Helper Macros (public, ATS-safe) --------
% Use public macro names (no @) to avoid needing \makeatletter.
\newcommand{\cvname}[1]{\def\CVName{#1}}
\newcommand{\cvrole}[1]{\def\CVRole{#1}}
\newcommand{\cvcontact}[1]{\def\CVContact{#1}}
% Granular contact fields
\newcommand{\cvphone}[1]{\def\CVPhone{#1}}
\newcommand{\cvemail}[1]{\def\CVEmail{#1}}
\newcommand{\cvaddress}[1]{\def\CVAddress{#1}}
\newcommand{\cvgithub}[1]{\def\CVGithub{#1}}
\newcommand{\cvwebsite}[1]{\def\CVWebsite{#1}}
\newcommand{\cvsummary}[1]{\def\CVSummary{#1}}

% Provide defaults
\cvname{Your Name}
\cvrole{Professional Title / Target Role}
\cvcontact{City, ST | (555) 555-5555 | professional.email@example.com | LinkedIn: linkedin.com/in/username | GitHub: github.com/username}
\cvphone{+000 0000000}
\cvemail{your.email@example.com}
\cvaddress{City, Country}
\cvgithub{https://github.com/username}
\cvwebsite{https://portfolio.example.com}
\cvsummary{Impact-focused professional with X+ years of experience ... (2-3 concise, keyword-rich sentences highlighting scope, strengths, domains, and outcomes.)}

% Section heading macro (simple for ATS parsing)
\newcommand{\cvsection}[1]{%
  \vspace{3pt}\par\textbf{\large #1}\par\vspace{1.5pt}\hrule height 0.6pt \vspace{2pt}
}

% Header macro - name and role on same line, contact below
\newcommand{\makecvheader}{%
  \begingroup
  \par\vspace*{0pt}
  {\Large\textbf{\MakeUppercase{\CVName}}\hfill\Large\textbf{\CVRole}}\par
  \vspace{1pt}
  \endgroup
}

% Contact information - compact single line
\newcommand{\makecontactinfo}{%
  {\small
    \CVPhone~\textbar~\href{mailto:\CVEmail}{\CVEmail}~\textbar~\CVAddress~\textbar~\href{\CVGithub}{GitHub}~\textbar~\href{\CVWebsite}{Portfolio}%
  }\par\vspace{2pt}
}

% Internship environment - tight spacing for 1 page
% Usage: \begin{internship}{Title}{Company}{Location}{Dates}{Company description}
%           \item Task bullet ...
%       \end{internship}
\newenvironment{internship}[5]{%
  \par\textbf{#1}\hfill{\footnotesize\textit{#4}}\\[-2pt]
  #2\\[-5pt]
  \ifx&#5&\else{\footnotesize\textcolor{gray}{#5}}\\[-6pt]\fi
  \begin{itemize}[leftmargin=8pt,itemsep=0pt,topsep=0pt]
}{\end{itemize}\vspace{2.5pt}}

% Entry macro: Title | Company | Location | Dates
% Usage: \cventry{TITLE}{COMPANY}{LOCATION}{DATES}{CONTENT}
\newcommand{\cventry}[5]{%
  \textbf{#1} \hfill #4\\
  \textit{#2, #3}\\[-4pt]
  #5\vspace{6pt}
}

% Compact bullet alias (alternative inline bullet list if ever needed)
\newcommand{\cvbullet}[1]{\item #1}

% Keywords environment (comma-separated plain text for ATS)
\newenvironment{cvkeywords}{%
  \vspace{4pt}\noindent\begin{minipage}{\textwidth}\textbf{Keywords: }\begingroup\raggedright\small
}{\par\endgroup\end{minipage}\vspace{4pt}}

% Date range helper (ensures consistent style)
\newcommand{\daterange}[2]{#1 -- #2}

% (Removed custom \endash macro to avoid redefinition warnings; use -- directly for en-dash)

% Safety: ensure no widow lines at page start if possible
\clubpenalty=10000
\widowpenalty=10000
\hyphenpenalty=10000
\exhyphenpenalty=10000

% Page style: no headers/footers and no page numbers
\pagestyle{empty}

% Single page bias: tighten a bit if near overflow (user can adjust)
% Uncomment to shrink vertical space a little more:
% \setlength{\parskip}{1pt}
% \setlist[itemize]{itemsep=1pt, topsep=1pt}

% End preamble macros

% Summary macro (renders the CV summary text if provided)
\newcommand{\makecvsummary}{{\small \CVSummary}\par}

% Defer metadata that uses user-provided macros until after they are defined
% so the PDF properties match the actual name/role/keywords.
\AtBeginDocument{%
  \hypersetup{
    pdfauthor={\CVName},
    pdftitle={\CVName\space — \CVRole},
    pdfsubject={Resume / CV},
    pdfkeywords={Software, Full-Stack, React, NestJS, Spring Boot, JavaScript, TypeScript, Python, MongoDB, SQL, DevOps, CI/CD, Docker, Jenkins, RAG}
  }
}


\cvname{Yosr Ben Nagra}
\cvrole{Web Developer}
\cvcontact{Ariana, Tunisia}
\cvphone{+216 53916040}
\cvemail{yosrbennagra@gmail.com}
\cvaddress{Ariana, Tunisia}
\cvgithub{https://github.com/YosrBennagra}
\cvwebsite{https://portfolio-yosr.vercel.app/en}

\begin{document}
\makecvheader
\makecontactinfo

\vspace{12pt}

\noindent Dear Canonical Hiring Team,

I'm excited to apply for the Web Developer position at Canonical. Your mission to make open source software available to people everywhere resonates deeply with me. As someone who's built web applications that serve diverse users, I understand the impact of creating accessible, high-quality software, and the opportunity to contribute to Ubuntu and Canonical's platform is genuinely inspiring.

I love building modern web applications with React and TypeScript. In my Collaborative Document Platform project, I created a Notion-like application where users collaborate in real-time through WebSockets. I built responsive UI components that work seamlessly across devices, implemented proper state management with Context API, and wrote comprehensive tests with Jest. The project taught me to think carefully about component architecture, performance optimization, and user experience, exactly the kind of work I see in Canonical's web portfolio.

At IronByte, I developed an educational platform with React and TypeScript, working closely with designers to implement intuitive interfaces for assignment submission and scheduling features. At Ooredoo, I used Angular to build an internal communication app with real-time chat, which required careful attention to responsive design and cross-browser compatibility. These experiences taught me to value design as much as code, collaborate effectively with UX and visual designers, and ensure the standard of output remains high across the entire product.

I'm passionate about web standards and constantly learning. I keep up with the latest CSS techniques, understand accessibility considerations (semantic HTML, ARIA, keyboard navigation), and think about SEO from the start. I value performance in complex user interfaces. I've optimized React apps by implementing lazy loading, memoization, and efficient state management. I use Git extensively for version control and collaboration, participate actively in code reviews, and believe in maintaining clean, well-documented code.

What excites me about this role is the opportunity to work on high-fidelity websites and web apps that impact millions of users worldwide. Contributing to Vanilla Framework, collaborating with a distributed team across 75+ countries, and helping shape the web presence of one of the most important open source projects: that's the kind of meaningful work I'm looking for. I'm based in Tunisia (EMEA timezone), which aligns perfectly with your requirements.

I have experience with server-side technologies too. I've built Python Flask backends and worked with Node.js/NestJS, which helps me understand the full stack and communicate effectively with backend developers. I'm comfortable working remotely and collaborating asynchronously through Git, GitHub, and other tools. I'm curious about technology, eager to learn about Linux desktop technologies and Canonical's ecosystem, and ready to embrace the challenge of developing for a broad, global audience.

I'm someone who genuinely loves what I do. I'm always looking for opportunities to improve my skills, I value both the design and the code equally, and I'm excited to show off what I'm working on while learning from talented colleagues. Joining Canonical would be a step into the future: working at the forefront of the global move to open source, contributing to software that changes the world.

Thank you for considering my application. I'd love to discuss how my React/TypeScript experience and passion for web standards can contribute to Canonical's mission.

\vspace{6pt}
\noindent Best regards,\\
\textbf{Yosr Ben Nagra}

\end{document}
